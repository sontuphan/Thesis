\chapter{Những công trình liên quan}
Như đã nhắc tới trước đó, các công trình về dự đoán xu hướng giá trị Bitcoin 
hầu như chưa có hoặc chưa được công khai vì thế mà việc tiếp cận chính xác vấn 
đề là điều không thể. Thay vào đó chúng ta sẽ đi sử dụng các vấn đề liên quan 
khác như là dự đoán xu hướng giá trị vàng và dự đoán xu hướng giá trị cổ phiếu.
Hai công trình cụ thể được tham khảo trong luận văn là:
\begin{enumerate}
\item Predicting Gold Prices - Megan Potoski \cite{PredictingGoldPrices}
\item Machine Learning in Stock Price Trend Forecasting - Yuqing Dai \& 
Yuning Zhang \cite{StockPriceTrendForecasting} 
\end{enumerate}
Ở bài báo thứ nhất - Predicting Gold Prices - đã đề cập đến hai giải thuật phân lớp 
là SVM và LR. Trong đó, vì va vấp với vấn đề mất cân đối trong tập 
dữ liệu (nhãn positive lớn hơn rất nhiều sao với nhãn negative) nên SVM chỉ được 
đề cập như một phép so sánh và không được sử dụng trong quá trình giải quyết vấn 
đề chính. Thay vào đó, LR được sử dụng để giải quyết bài toán 
phân lớp với kết quả khá khả quan.\\\\
LR (Optimal Feature Set):
\begin{table}[h]
\centering
\begin{tabular}{ |c|c| }
\hline
Precision & 69.90\% \\
\hline
Recall & 72.31\% \\
\hline
Accuracy & 69.30\% \\
\hline
\end{tabular}
\caption{Bảng đánh giá - Predicting Gold Prices}
\end{table}\\
Ở đây, ta nhận thấy bài báo sử dụng ba tham số đánh giá, chưa vội quan tâm đến 
ý nghĩa từng tham số ta có thể hiểu rằng các tham số này càng cao thì tương 
đương với giái thuật càng được xem là tốt. Chi tiết ba tham số này sẽ được nhắc 
đến ở phần Nền tảng lý thuyết.\\\\
Bước qua bài báo thứ hai - Machine Learning in Stock Price Trend Forecasting - 
nhóm tác giả đã sử dụng bốn giải thuật đó là:
\begin{itemize}
\item GDA
\item LR
\item SVM 
\item QDA
\end{itemize}
Kết quả đánh giá của 4 giải thuật được nhóm tác giả trình bày:
\begin{table}[h]
\centering
\begin{tabular}{ |c|c|c|c|c| }
\hline
Model & LR & GDA & QDA & SVM \\
\hline
Accuracy & 44.5\% & 46.4\% & 58.2\% & 55.2\% \\
\hline
\end{tabular}
\caption{Bảng đánh giá - Machine Learning in Stock Price Trend Forecasting }
\end{table}\\
Thật sự kết quả cho ra không tốt so với bài báo thứ nhất và tham số đánh giá 
chỉ sử dụng một tham số đó là Accuracy, chúng ta không thể dựa vào đó để đánh giá 
một cách toàn diện về độ hiệu quả của giải thuật. Nhưng riêng trong công trình 
này, nhóm tác giả có nêu ra Next-Day Model nhằm dự đoán xu hướng giá cổ phiếu 
trong ngày tiếp theo và có vẻ khá tương đồng với vấn đề đặt ra trong phạm vi 
luận văn này.\\\\
Tổng quan qua hai công trình và tham khảo một số công trình khác, nhận thấy 
đa số các hướng tiếp cận đều đi theo một phương pháp tổng quát chung, nó bao gồm các bước 
cơ bản như:
\begin{enumerate}
\item Xây dựng không gian vector thuộc tính phù hợp với tính chất bài toán
\item Sử dụng các giải thuật phân lớp điển hình trong Máy học như là 
SVM, LR ...
\item Đánh giá giải thuật bằng các tham số Accuracy, Recall, Precision.
\end{enumerate}
Từ những đục kết trên, bản thân nhận thấy các bước trên cũng chính là phương pháp 
nên dùng để tiếp cận đề tài. Ngoài ra, nhận thấy ở hai công trình trên chưa 
hề sử dụng một giải thuật rất được phổ biến hiện nay, nó nổi lên như một đại 
diện của Deep Learning đó là Multilayer Neural Network. Do đó mà luận văn này sẽ sử dụng 
Multilayer Neural Network như là một giải thuật chính trong quá trình so sánh 
và đánh giá so với các giải thuật phân lớp khác.