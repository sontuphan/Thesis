\chapter{Giới thiệu đề tài}
\section{Tính cấp thiết của đề tài}
BITCOIN - một loại tiền mã hóa (hay tiền điện tử) được xuất hiện lần đầu tiên 
vào năm 2009 bởi Satoshi Nakamoto \cite{Bitcoin}, với những đặc tính ưu việt hơn cả tiền tệ 
truyền thống hiện nay khiến cho sự tăng lên nhanh chóng về giá trị. Nhận thấy 
được sức mạnh của tiền mã hóa có thể sẽ là tương lai của kinh tế và chính trị 
nên việc hiểu rõ cũng như đầu tư vào BITCOIN là việc đáng để suy ngẫm.\\\\
Hiển nhiên, đối với nước ta BITCOIN là rất mới và việc đầu tư là hết sức rủi ro
khi không có nền tảng kiến thức và kinh nghiệm đầu tư. Nhận thấy vấn đề này, 
bản thân đã đặt ra vấn đề``Tại sao không tạo ra một công cụ để cho nhà đầu tư 
có thể dựa vào như một yếu tố tham khảo tin cậy''.\\\\
Đồng thời, trong lĩnh vực Công nghệ thông tin nói riêng, Machine Learning đang là 
nền tảng cho hàng loạt các sản phẩm công nghệ mang tính dự đoán thông minh, ngoài 
ra còn ứng dụng trong các lĩnh vực về trí thông minh nhân tạo, xử lý ngôn ngữ 
tự nhiên... và điều đó đang đi đúng với mục tiêu của vấn đề được đưa ra trong phạm 
vi luận văn này.

\section{Đặc tả đề tài}
Trên một sàn giao dịch tiền mã hóa điển hình, quá trình mua bán BITCOIN được chia ra 
thành các giai đoạn thời gian và được gọi là phiên giao dịch. Một phiên giao dịch 
được diễn tả bởi các giá trị điển hình như sau:
\begin{itemize}
\item Giá mở phiên: giá bán (mua) BITCOIN của (các) giao dịch ngay tại thời 
điểm mở phiên.
\item Giá đóng phiên: giá bán (mua) BITCOIN của (các) giao dịch tại thời điểm 
kết thúc phiên.
\item Giá cao nhất: giá bán (mua) BITCOIN cao nhất của giao dịch trong khoảng 
thời gian mở phiên đến kết thúc phiên.
\item Giá thấp nhất: giá bán (mua) BITCOIN thấp nhất của giao dịch trong khoảng 
thời gian mở phiên đến kết thúc phiên.
\end{itemize}
Thời gian của một phiên giao dịch thường được chọn là 5 phút, 30 phút, 1 tiếng, 2 tiếng, 
4 tiếng hoặc 1 ngày, ... 
Trong phạm vi luận văn chúng ta chọn thời gian một phiên giao dịch là 30 phút.
Ở đây, bài toán là đi dự đoán giá trị BITCOIN trong phiên tiếp theo sẽ tăng 
hay giảm so với phiên hiện tại. Cụ thể, gọi n là phiên hiện tại và n(close) 
là giá đóng phiên hiện tại, n+1 là phiên tiếp theo và n+1(close) là giá đóng 
phiên tiếp theo. Nếu n+1(close) > n(close) thì giá tăng (Up), ngược lại thì 
giá giảm (Down).\\\\
Sau khi cụ thể được yêu cầu bài toán, ta sẽ đi đặc tả hướng tiếp cận giải quyết 
vấn đề. Machine Learning là lựa chọn của luận văn này, cụ thể phương pháp giải quyết 
sẽ sử dụng giải thuật phân lớp để dự đoán nhãn của phiên giao dịch sẽ là Up hay Down.

\section{Mục tiêu của đề tài}
Vấn đề cơ bản của việc đầu tư là lợi nhuận, bám sát với mục tiêu này phương hướng 
đề ra sẽ đi giải quyết bài toán cụ thể như sau:\\\\
Sử dụng USD để mua/bán BITCOIN, với mỗi phiên giao dịch là 30 phút, chúng 
ta sẽ đi dự đoán giá trị BITCOIN trong phiên tiếp theo sẽ tăng hay giảm - bài
toán phân lớp trong Machine Learning.\\\\
Để thực hiện được điều đó chúng ta cần vạch ra những bước đi cụ
thể để hiện thực mục tiêu:
\begin{itemize}
  \item Thu thập, xử lí dữ liệu BITCOIN.
  \item Áp dụng các giải thuật phân lớp vào tập dữ liệu có được.
  \item Đánh giá trên lý thuyết hệ thống.
  \item Vận hành, khảo sát và đánh giá hệ thống trên thực tế.
  \item Xây dựng, hoàn thiện sản phẩm.
\end{itemize} 
Sản phẩm hoàn thiện mà người dùng được sử dụng sẽ là một Ứng dụng nền Web cung 
cấp các thông tin, quan điểm để tham khảo cho việc đầu tư.
\section{Phương pháp thực hiện đề tài}
Vì bài toán dự đoán về xu hướng giá trị BITCOIN hầu như chưa có bất kì 
công trình hoặc bài báo nào được công bố công khai (theo tìm hiểu của cá nhân) 
nên việc phải tham khảo các hướng giải quyết đã từng có là bất khả thi. Thay vào 
đó chúng ta sẽ đi tham khảo các bài báo, công trình có mức độ liên quan khá cao 
như dự đoán xu hướng giá vàng và giá cổ phiếu - các tài liệu này được dẫn tại 
phần tài liệu tham khảo.
Từ những kinh nghiệm của các bài báo, bản thân sẽ đúc kết một vài phương pháp 
tổng quát, từ đó áp dụng ngược trở lại cho vấn đề dự đoán xu hướng giá trị 
BITCOIN.\\\\
Đồng thời, ngoài việc tham khảo các công trình liên quan, bản thân còn phải 
sử dụng chính những kinh nghiệm về khai phá dữ liệu và kiến thức Machine Learning, 
để áp dụng vào nhằm đem lại kết quả tốt nhất. Việc tìm ra lời giải tốt nhất sẽ 
tiến hành theo phương pháp so sánh các giải thuật, chúng ta sẽ đi chạy các giải 
thuật phân lớp khác nhau từ đó đánh giá xem giải thuật nào là tốt hơn và từ đó 
sẽ tập trung tối ưu cho giải thuật đó.\\\\
Sản phẩm hoàn thiện là sản phẩm đã được chạy và khảo nghiệm trên thực tế, vì vậy 
sau khi xây dựng hoàn chỉnh, hệ thống sẽ được chạy thực tế và đánh giá kết quả 
trong một khoảng thời gian.
\section{Bố cục luận văn}
Để phục vụ tốt cho việc phát triển sau này, bố cục luận văn sẽ được trình bày 
theo hướng diễn dịch và được chia thành các phần nhỏ để người đọc có thể nắm 
bắt nội dung.\\\\
Trước hết, chúng ta sẽ đi tìm hiểu qua công trình liên quan nhằm hiểu được công việc 
chúng ta sẽ làm là gì? Và những hướng giải quyết tổng quát đã được sử dụng ra 
sao?\\\\
Sau đó, phần Nền tảng lý thuyết sẽ trang bị các kiến thức về Đại số, Giải tích và Kinh tế để 
phục vụ cho việc đọc hiểu nội dung các chương sau, đặc biệt là phục vụ cho quá 
trình phân tích giải thuật phân lớp trong Machine Learning - cụ thể là sử dụng 
Multilayer Neural Network để phân lớp.\\\\
Cuối cùng, thu thập dữ liệu và khai phá dữ liệu cho phù hợp với giải thuật, 
chạy giải thuật, đánh giá giải thuật và hiện thực sản phẩm.
