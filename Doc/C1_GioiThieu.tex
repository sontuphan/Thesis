\chapter{Giới thiệu đề tài}
\section{Tính cấp thiết của đề tài}
Bitcoin - một hệ thống tiền mã hóa (hay tiền điện tử) được xuất hiện lần đầu tiên 
vào năm 2009 bởi Satoshi Nakamoto \cite{BitcoinPaper}, với những đặc tính ưu việt hơn cả tiền tệ 
truyền thống hiện nay đã khiến cho sự tăng lên nhanh chóng về giá trị. Nhận thấy 
được sức mạnh của tiền mã hóa có thể sẽ là tương lai của kinh tế và chính trị 
nên việc hiểu rõ cũng như đầu tư vào Bitcoin là việc đáng để quan tâm.\\\\
Trong giai đoạn hiện nay, đối với nước ta, Bitcoin là một khái niệm mới vì thế 
mà việc đầu tư khi chưa có nền tảng kiến thức hoặc kinh nghiệm đầu tư là hết 
sức rủi ro. Nhận thấy vấn đề này, bản thân đã đặt ra vấn đề ``Tại sao không 
tạo ra một công cụ để cho nhà đầu tư có thể dựa vào như một yếu tố tham khảo 
tin cậy?''.\\\\
Đồng thời, trong lĩnh vực công nghệ thông tin nói riêng, Học máy đang là 
nền tảng cho hàng loạt các sản phẩm công nghệ mang tính dự đoán thông minh, ngoài 
ra còn ứng dụng trong các lĩnh vực về trí thông minh nhân tạo, xử lý ngôn ngữ 
tự nhiên... và điều đó đang đi đúng với mục tiêu của vấn đề được đưa ra trong phạm 
vi bài dự thi này.

\section{Đặc tả đề tài}
Trên một sàn giao dịch tiền mã hóa điển hình, quá trình mua bán BTC được chia ra 
thành các giai đoạn thời gian và được gọi là phiên giao dịch. Một phiên giao dịch 
được diễn tả bởi các giá trị điển hình như sau:
\begin{itemize}
\item Giá mở phiên: giá bán (mua) BTC của (các) giao dịch ngay tại thời 
điểm mở phiên.
\item Giá đóng phiên: giá bán (mua) BTC của (các) giao dịch tại thời điểm 
kết thúc phiên.
\item Giá cao nhất: giá bán (mua) BTC cao nhất của giao dịch trong khoảng 
thời gian mở phiên đến kết thúc phiên.
\item Giá thấp nhất: giá bán (mua) BTC thấp nhất của giao dịch trong khoảng 
thời gian mở phiên đến kết thúc phiên.
\end{itemize}
Thời gian của một phiên giao dịch thường được chọn là 5 phút, 30 phút, 1 tiếng, 2 tiếng, 
4 tiếng hoặc 1 ngày, ... 
Trong phạm vi đề tài chúng ta chọn thời gian một phiên giao dịch là 30 phút.\\\\
Vậy, bài toán cần giải quyết là đi dự đoán giá trị BTC trong phiên tiếp theo sẽ tăng 
hay giảm so với phiên hiện tại. Cụ thể, gọi $n$ là phiên hiện tại và $n_{close}$ 
là giá đóng phiên hiện tại, $(n+1)$ là phiên tiếp theo và $(n+1)_{close}$ là giá đóng 
phiên tiếp theo. Nếu $(n+1)_{close} > n_{close}$ thì giá tăng - $Up$, ngược lại thì, 
$(n+1)_{close} \leq n_{close}$ thì giá giảm - $Down$.\\\\
Sau khi cụ thể được yêu cầu bài toán, ta sẽ đi đặc tả hướng tiếp cận giải quyết 
vấn đề. Học máy là lựa chọn của đề tài này, cụ thể phương pháp giải quyết 
sẽ sử dụng giải thuật phân lớp để dự đoán nhãn của phiên giao dịch sẽ là $Up$ 
hay $Down$.

\section{Mục tiêu của đề tài}
Vấn đề cơ bản của việc đầu tư là lợi nhuận, bám sát với mục tiêu này phương hướng 
đề ra sẽ đi giải quyết bài toán cụ thể như sau.\\\\
Sử dụng USD để mua/bán BTC, với mỗi phiên giao dịch là 30 phút, chúng 
ta sẽ đi dự đoán giá trị BTC trong phiên tiếp theo sẽ tăng hay giảm - bài
toán phân lớp trong Học máy.\\\\
Để thực hiện được điều đó chúng ta cần vạch ra những bước đi cụ
thể để hiện thực mục tiêu:
\begin{itemize}
  \item Thu thập, xử lí dữ liệu BTC.
  \item Áp dụng các giải thuật phân lớp vào tập dữ liệu có được.
  \item Đánh giá trên lý thuyết hệ thống.
  \item Vận hành, khảo sát và đánh giá hệ thống trên thực tế.
  \item Xây dựng, hoàn thiện sản phẩm.
\end{itemize} 
Sản phẩm hoàn thiện mà người dùng được sử dụng sẽ là một ứng dụng nền web cung 
cấp các thông tin về dự đoán và các thông số thống kê dùng để tham khảo cho việc 
đầu tư.
\section{Phương pháp thực hiện đề tài}
Các công trình hoặc bài báo liên quan đến đề tài dự đoán BTC bằng Học máy hầu 
như mới chỉ xuất hiện một vài năm gần đây và chỉ một số ít trong các nghiên cứu 
này được công bố một cách công khai. Vì những giới hạn này, quá trình nghiên 
cứu và phân tích đề tài của bài dự thi này ngoài tham khảo những công trình có 
liên quan trực tiếp đến đề tài, sẽ còn tham khảo các công trình nghiên cứu khác 
có mức độ liên quan một các tương đối. Cụ thể như các công trình liên 
quan đến sử dụng Học máy trong dự đoán giá trị của vàng hoặc cổ phiếu.\\\\
Từ những kinh nghiệm của các công trình này, bản thân sẽ đúc kết một phương pháp tổng 
quát, từ đó áp dụng trở lại cho vấn đề dự đoán xu hướng giá trị BTC.\\\\
Đồng thời, ngoài việc tham khảo các công trình liên quan, bản thân còn vận dụng 
những kinh nghiệm cá nhân về khai phá dữ liệu và kiến thức Học máy, để áp dụng 
vào quá trình nghiên cứu nhằm đem lại kết quả tốt nhất. Việc tìm ra lời giải 
tốt nhất sẽ tiến hành theo phương pháp so sánh kết quả giữa các giải thuật, 
chúng ta sẽ đi chạy các giải thuật phân lớp khác nhau từ đó đánh giá xem giải 
thuật nào là tốt hơn và sẽ tập trung tối ưu cho giải thuật đó.\\\\
Sản phẩm hoàn thiện là sản phẩm đã được chạy và khảo nghiệm trên thực tế, vì vậy 
sau khi xây dựng hoàn chỉnh, hệ thống sẽ được chạy thực tế và đánh giá kết quả 
trong một khoảng thời gian.
\section{Bố cục đề tài}
Để phục vụ tốt cho việc phát triển sau này, bố cục đề tài sẽ được trình bày 
theo hướng diễn dịch và được chia thành các phần nhỏ để người đọc có thể nắm 
bắt nội dung.\\\\
Trước hết, chúng ta sẽ đi tìm hiểu qua các công trình liên quan nhằm hiểu được 
công việc chúng ta sẽ làm là gì, và những hướng giải quyết tổng quát đã được 
sử dụng ra sao.\\\\
Sau đó, phần nền tảng lý thuyết sẽ đề cập đến các kiến thức liên quan đến Bitcoin, một số 
khái niệm về tài chính, cũng như lý thuyết giải thuật MNN dùng cho phân lớp để 
phục vụ cho việc đọc hiểu nội dung các chương sau, đặc biệt là phục vụ cho quá 
trình phân tích giải thuật phân lớp trong Học máy.\\\\
Cuối cùng, thu thập dữ liệu và khai phá dữ liệu cho phù hợp với giải thuật, 
chạy giải thuật, đánh giá giải thuật và hiện thực sản phẩm.
