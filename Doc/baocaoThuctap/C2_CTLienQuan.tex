\chapter{Những công trình liên quan}
Ngày nay, vấn đề bảo mật dữ liệu đang rất cần thiết, số lượng các cuộc xâm nhập
tấn công ngày càng gia tăng, cách thức tấn công cũng hết sức đa dạng và các cuộc
tấn công với cách thức chưa xuất hiện bao giờ ngày càng nhiều. Lấy ý tưởng từ hệ
thống phát hiện xâm nhập, từ rất sớm, các nhà khoa học đã mong muốn xây dựng một
hệ thống kết hợp với học máy để có thể nhận biết các xâm nhập bất thường mà
không cần định nghĩa hành vi xâm nhập một cách cụ thể.\\\\  
 Một trong những
công trình gần đây có thể kể đến là “Developing a high-accuracy cross platform
Host-Based Intrusion Detection System capable of reliably detecting zero-day
attacks” – Gideon Creech. Tổng quan về công trình này, đây là một bài luận văn,
xây dựng hệ thống HIDS dựa trên việc học từ tập dữ liệu về systemcall có tên là
ADFA. Trong công trình này, sử dụng hai hướng tiếp cận:
\begin{itemize}
  \item Sử dụng giải thuật học máy One-Class SVM với độ chính xác cao, 15\%
   với tỉ lệ báo sai và 80\% với tỉ lệ báo đúng. Nhưng, vấn đề ở đây là giải
   thuật mất khá nhiều thời gian dành cho việc chạy giải thuật và gần như
   không thực tế trong công nghiệp.
   \item Sử dụng giải thuật học máy K-mean, vì giải thuật này tương đối đơn
   giản nên việc học là rất nhanh, cho ra kết quả trong thời gian ngắn. Nhưng,
   đánh đổi ở đây là độ chính xác, độ chính xác cho ra lại quá thấp và thiếu
   chính xác, điều này cũng dẫn đến việc không thể để hiện thực trong công nghiệp. Tỉ lệ báo đúng là 60\% và tỉ lệ báo sai là 20\%.
\end{itemize}
Nhìn nhận hướng tiếp cận này đã được giải quyết tốt bởi một chuyên gia – Creech
– nhưng vẫn chưa thể áp dụng được vào công nghiệp mà chỉ mang tính chất nguyên
cứu, học thuật nên nhóm quyết định thay đổi hướng tiếp cận với bài toán ban
đầu.\\\\ 
Ở đây, bài toán của chúng ta là xây dựng một hệ thống phát hiện
xâm nhập mà không cần định nghĩa các hành vi xâm nhập một cách rõ ràng bằng cách
dựa vào cơ sở học máy.\\\\ 
Hướng tiếp cận cổ điển gồm các bước:
\begin{enumerate}
  \item Thu thập dữ liệu mẫu.
  \item Sử dụng giải thuật gom cụm (unsupervised) để phân loại (không nhãn)
  đối với dữ liệu mẫu.
  \item Dùng chuyên gia để gán nhãn.
  \item Dùng giải thuật phân lớp (supervised) để phân loại (có nhãn) đối với dữ
  liệu đã được chuyên gia xử lý.
  \item Kiểm tra, đánh giá, tối ưu và áp dụng mô hình cuối cùng vào hệ thống
  HIDS.
\end{enumerate}
  Hướng tiếp cận của nhóm:
\begin{enumerate}
  \item Thu thập dữ liệu mẫu.
  \item Sử dụng giải thuật gom cụm (unsupervised) để phân loại (không nhãn) đối
  với dữ liệu mẫu.
  \item Dùng chuyên gia gán nhãn.
  \item Đánh giá, tối ưu và tạo ra mô hình học máy.
\end{enumerate}
  Trong mô hình cổ điển, ta thấy các giải thuật gom cụm chủ yếu chỉ để giúp
chuyên gia trong việc giảm tải khối lượng phải đánh nhãn, việc tạo ra mô hình
học máy vẫn là nằm ở giải thuật phân lớp. Còn trong mô hình nhóm đề xuất, nhóm đã đưa
giải thuật gom cụm thành giải thuật chính và sử dụng vào việc tạo ra mô hình học
máy.\\\\ 
Nhóm cũng nhận thấy rằng, với hướng tiệp cận này chi phí phát
triển cho toàn bộ quá trình được rút ngắn đi rất nhiều. Đồng thời, việc sử dụng 
cùng một giải thuật gom cụm ở cả quá trình tiền xử lý và quá trình học cũng giúp
hạn chế thay đổi tính chất cụm và có thể nâng cao độ chính xác ở một mức độ
nào đó.\\\\ 
Vì hướng tiếp cận này tương đối mới nên nhóm tập trung vào việc
tìm ra một “đường học” có thể cân bằng được giữa tốc độ và độ chính
xác.\\\\ 
Với tất cả những sự nhìn nhận trên, nhóm quyết định xây dựng một
hướng tiếp cận mới và cố gắng phát huy những ưu điểm đang có của hướng tiếp cận
này và hạn chế mắc phải những nhược điểm mà hướng tiếp cận cổ điển đã mắc trước
đó.   
