\chapter{Giới thiệu đề tài}
\section{Tính cấp thiết của đề tài}
Machine learning là một lĩnh vực của trí tuệ nhân tạo liên quan đến việc nghiên
cứu và xây dựng các kĩ thuật cho phép các hệ thống “học” tự động từ dữ liệu để
giải quyết những vấn đề cụ thể. Ví dụ như các máy có thể “học” cách phân loại
thư điện tử xem có phải thư rác (spam) hay không và tự động xếp thư vào thư mục
tương ứng. Thế nhưng machine learning đã đi xa hơn những khái niệm ban đầu rất
nhiều.\\\\ 
Chúng em chọn đề tài này vì nó khá mới mẻ ở Việt Nam và nhận
thấy những lợi ích mà học máy đem lại khi giải quyết những vấn đề thực tiễn
trong cuộc sống, ứng dụng trong nhiều lĩnh vực của đời sống xã hội. Bên cạnh đó,
cơ hội việc làm về lĩnh vực học máy hay trí tuệ nhân tạo ở Việt Nam hiện nay là
khá lớn, chúng em muốn nghiên cứu đề tài không chỉ tìm hiểu thêm cái mới mà còn
để định hướng cho tương lai khi ra trường.\\\\ 
Hiện nay đã có một số lĩnh
vực áp dụng thành công kĩ thuật học máy để giải quyết công việc như: Bệnh án
điện tử, dự đoán thị trường chứng khoán, dự báo thời tiết, \dots. Ngoài ra, việc
áp dụng học máy trong bảo mật thông tin đang là đề tài mới ở Việt Nam mà nếu áp
dụng thành công học máy thì sẽ giúp các doanh nghiệp Việt Nam dự báo được các
cuộc tấn công an ninh mạng từ đó có giải pháp bảo vệ tài sản thông tin tốt hơn.
\section{Mục tiêu của đề tài}
Trong lĩnh vực Học máy có các phương pháp học sau: Học có giám sát (supervised
learning), Học không có giám sát (unsupervised learning), Học bán giám sát
(semi-supervised learning), Học tăng cường (reinforcement learning). Đề tài của
nhóm chúng em sẽ tập trung vào kĩ thuật học không giám sát cụ thể là tìm hiểu
các kĩ thuật gom cụm trong học máy từ đó áp dụng vào một ví dụ cụ thể đó là phân
cụm một sự kiện xảy ra trong máy tính có phải là một cuộc tấn công an ninh mạng
hay không. Để thực hiện được điều đó nhóm chúng em cần vạch ra những mục tiêu cụ
thể để thực hiện đề tài:
\begin{itemize}
  \item Tìm hiểu các kĩ thuật gom cụm phổ biến
  \item Tìm hiểu giải thuật thu giảm thuộc tính của ma trận trước khi đưa học
  máy vào để phân cụm
  \item Tìm hiểu các công cụ áp dụng học máy và vận dụng vào một tập dữ liệu mẫu
  \item Khảo sát và đánh giá độ hiệu quả của từng giải thuật học máy.
  \item Tìm hiểu framework mã nguồn mở về bảo mật thông tin OSSEC HIDS.
\end{itemize} 
  Nhóm chúng em đã tìm hiểu được ba giải thuật gom cụm là Kmeans, EM và
SOM, sử dụng giải thuật PCA, SVD để thu giảm số chiều của ma trận thuộc tính, áp
dụng các kiến thức đó vào việc phân cụm cuộc tấn công an ninh mạng, đánh giá kết quả và
sử dụng một vài chức năng của OSSEC HIDS.\\\\ 
Trong giới hạn kiến thức tìm hiểu, nhóm dừng lại ở việc tìm hiểu giải thuật, sử
dụng công cụ có sẵn để áp dụng các giải thuật gom cụm vào tập dữ liệu. Nhóm chưa
áp dụng các kĩ thuật học máy vào OSSEC HIDS để xây dựng luật mà chỉ dừng lại ở
việc tìm hiểu một số chức năng có sẵn của công cụ. Đồng thời kết quả của việc
phân cụm của tập dữ liệu chưa đạt được hiệu quả như mong muốn.
\section{Phương pháp thực hiện đề tài}
Mục tiêu đề ra ở phần trên gồm 2 nội dung chính: nghiên cứu về học máy và tìm
hiểu về công cụ phát hiện xâm nhập.\\\\ 
Với nội dung đầu, nhóm tiếp cận theo hướng tìm các tài liệu liên quan, nghiên
cứu sau về lý thuyết từng giải thuật, hiểu rõ bản chất, và đánh giá cơ bản. Sau
đó tìm tập dữ liệu thực tế, sử dụng các công cụ để áp dụng những giải thuật
đã tìm hiểu vào tập dữ liệu, tiến hành đánh giá, và cải tiến giải thuật nếu
cần.\\\\ 
Với nội dung sau, nhóm tập trung vào tìm hiểu sâu các module có trong hệ thống.
Hiểu rõ hệ thống từ đó có thể sửa đổi theo nhu cầu đề tài.
\section{Bố cục}
Bố cục báo cáo được trình bày theo hướng để người đọc có thể dễ dàng tiếp
cận và nắm bắt vấn đề. Nên trong phần tiếp theo sẽ trình bày những công
trình liên quan và rút ra những kết luận về công trình này. Sau đó sẽ là nền
tảng về lý thuyết để người đọc hiểu được phương pháp tiếp cận, hiện thực, ý
tưởng của nhóm đưa ra và các kết quả đạt được. Phần tiếp theo là thiết kế, giải
pháp, ý tưởng dựa vào những kiến thức nền tảng trên. Cuối cùng nhóm sẽ kết luận lại những gì
đã đạt được, và đề ra hướng phát triển kế tiếp. 
