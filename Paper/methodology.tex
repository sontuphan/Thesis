\section{Thu thập dữ liệu và lựa chọn giải thuật}
\subsection{Thu thập dữ liệu}
\subsubsection{Nguồn dữ liệu}
Poloniex là một sàn giao dịch tiền mã hóa trực tuyến và có trụ sở tại Mỹ. Được 
thành lập vào tháng 1 năm 2014, khi đi vào hoạt động, Poloniex định hướng cung 
cấp một môi trường thương mại an toàn, đồng thời còn cung cấp các dữ liệu về 
thị trường như biểu đồ, bảng xếp hạng và các công cụ phân tích dữ liệu để hỗ 
trợ khách hàng. Ngoài ra, Poloniex còn cung cấp một số lượng dữ liệu liên quan 
đến các thống kê mua/bán của sàn giao dịch, các dữ liệu này được cung cấp thông qua API.\\\\ 
Nguồn dữ liệu được lấy thông qua API dữ liệu biểu đồ thị trường của sàn giao 
dịch Poloniex. Cụ thể API:
\begin{lstlisting}
https://poloniex.com/public?command=returnChartData&currencyPair=BTC_XMR&start=1405699200&end=9999999999&period=14400
\end{lstlisting}
Tập dữ liệu về các phiên giao dịch Bitcoin được thu thập từ ngày 20/2/2015 đến 
ngày 29/10/2016 và có tổng cộng 29634 mẫu (mỗi mẫu là đại diện của một phiên 
giao dịch).
Dữ liệu trả về là một mảng các phần tử JSON có dạng như sau:
\begin{lstlisting}
[
    {"date":1424372400,"high":225,"low":225,"open":225,"close":225,"volume":0.999999,"quoteVolume":0.00444444,"weightedAverage":225},
    {"date":1424374200,"high":225,"low":225,"open":225,"close":225,"volume":0,"quoteVolume":0,"weightedAverage":225},
    ...,
    {"date":1482575400,"high":900.79862142,"low":894.98864451,"open":900.79862142,"close":895.56277339,"volume":2427.10998126,"quoteVolume":2.70065724,"weightedAverage":898.71085649}
]
\end{lstlisting}
Sau khi thu thập, dữ liệu được tiền xử lý để loại bỏ các thông tin không được 
sử dụng trong quá trình phân tích và xây dựng giải thuật. Giá trị có khóa là 
$close$ là dữ liệu sẽ được sử dụng, các giá trị như: $date$, $high$, $low$, 
$open$, $volume$, $quoteVolume$, $weightedAverage$ sẽ được lược bỏ.
\subsubsection{Xây dựng dữ liệu luyện tập}
Gọi $S$ là đại diện cho một phiên giao dịch, các đặc trưng được xây dựng như 
sau:
\begin{itemize}
    \item 10 feature RDP: $\{ \: loop\{ RDP_1(S_{i+j})\}_i \: \}_j$ Với 
    $i \in [0:9], \: j \in [0:29634]$
    \item 1 feature SO. Với $ j \in [0:29625] $:\\
    \[
        \{ \%K_j = \frac{P(j+9)-L_{10}}{H_{10}-L_{10}} \}_j
    \]
    \item 1 feature ROC. Với $ j \in [0:29625] $:\\ 
    \[
        \{ ROC_{10}(j)= \frac{P(j+9) - P(j)}{P(j)} \}_j
    \]
\end{itemize}
Ở đây, chúng ta chọn mỗi vector đặc trưng được hình thành bởi 10 phiên giao 
dịch. Các giá trị SO và ROC đều được tính trong thời gian là 10 phiên giao dịch.
Sau khi đã có tập luyện tập (tập hợp các vector đặc trưng), ta cần nhãn - label 
để phân lớp tập luyện tập. Nhãn được định nghĩa như sau, nếu giá BTC ở phiên 
thứ 11 lớn hơn phiên thứ 10 thì nhãn sẽ là 1, ngược lại sẽ là 0 (Phiên 11 chính 
là phiên thứ 1 của nhóm 10 phiên liền sau nhóm 10 phiên hiện đang xét).\\
\[
    label_i = \bigg \{ _{0 \quad if \: P_i(10) \: \leq \: P_{i+1}(1)} ^{1 \quad if \: P_i(10) \: > \: P_{i+1}(1)}
\]
Kết quả dữ liệu luyện tập:
\begin{lstlisting}
0 0.06666666666666667 0.016666666666666666 0 0 0 0 0 0 0 0.08444444444444445 1 0
0.06666666666666667 0.016666666666666666 0 0 0 0 0 0 0 0 0.016666666666666666 1 0
...
0.002400968290280231 3.9565217470358025e-9 0.003910087158215986 -0.00045140752064857115 -0.0013329209110243738 0.0008262923440509297 0.018429551716218056 0.003236901952232515 -0.004644611078085971 -0.0016267310238298762 0.018318840535169866 0.7405268424216611 0
\end{lstlisting}
Mỗi hàng đại diện cho một vector đặc trưng và nhãn, chi tiết ở một vector đặc 
trưng 10 giá trị đầu sẽ là 10 đặc trưng $RDP$, 1 giá trị tiếp theo là đặc trưng 
$ROC$, 1 giá trị tiếp theo là đặc trưng $SO$ và 1 giá trị cuối cùng là nhãn.

\subsection{Lựa chọn giải thuật}
Bên cạnh chạy giải thuật MNN, chúng ta sẽ chạy các giải thuật khác nhằm so sánh 
và đánh giá giải thuật chính. Các giải thuật được chọn để so sánh với giải thuật 
chính: SVM (Support Vetor Machines), KNN (K-Nearest Neighbors), LR (Logistic Regression).\\\\
Để đánh giá mức độ ý nghĩa của từng giải thuật, chúng ta sẽ sử dụng 3 tham số 
đánh giá là Accuracy, Recall, Precision. Tập dữ liệu sẽ được chia ra thành hai phần:
\begin{itemize}
\item Tập huấn luyện: chiếm 7/10 tổng số dữ liệu, dùng để chạy trong quá trình 
học của giải thuật.
\item Tập đánh giá: chiếm 3/10 tổng số dữ liệu, dùng để chạy trong quá trình 
đánh giá giải thuật.
\end{itemize}
Mô hình đánh giá giải thuật được mô tả bằng cách, dùng tập huấn luyện để tạo 
ra mô hình học máy của các giải thuật, sau đó sử dụng tập đánh giá để làm đầu 
vào cho từng mô hình, kết quả dự đoán sẽ được so sánh với kết quả thực tế của 
từng vector đặc trưng đầu vào. Cụ thể, kết quả của 4 giải thuật được ghi nhận 
như sau:
\begin{table}[h]
\centering
\fontsize{8}{9}\selectfont
\begin{tabular}{ |c|c|c|c|c| }
\hline
 & KNN & LR & SVM & MNN \\
\hline
Accuracy & 62.93\% & 66.24\% & 66.40\% & 69.86\% \\
\hline
Precision & 44.69\% & 18.18\% & 0\% & 60.50\% \\
\hline
Recall & 43.62\% & 0.15\% & 0\% & 29.55\% \\
\hline
\end{tabular}
\caption{Bảng đánh giá}
\end{table}\\
Trước tiên theo bảng đánh giá, ta có các giải thuật LR và SVM cho kết quả Accuracy
là gần khoảng 66\%, nhưng khi nhìn vào chi tiết các giá trị Precision và Recall 
ta nhận thấy kết quả điều cho ra rất thấp. Kết quả này cho thấy giải thuật LR 
và SVM đều có số lần True Positive là xấp xỉ bằng 0, đồng nghĩa với việc các 
giải thuật này hầu như chỉ dự đoán kết quả là nhãn $Down$ cho tất cả trường hợp. 
Điều này hoàn toàn không có ý nghĩa trong dự đoán đầu tư.\\\\
Xét đến KNN và MNN, đối với KNN ta có thể thấy giải thuật có xu hướng cân bằng 
các giá trị Accuracy, Precision và Recall. Nhưng đối với MNN, giải thuật có xu 
hướng tối ưu hóa bộ thiêu chuẩn Accuracy và Precision. Vậy câu hỏi đặt ra ở đây 
là kết quả nào có giá trị đầu tư hơn?\\\\
Chú ý đến Recall, dựa theo định nghĩa thì Recall có thể hiểu nếu trong thực tế 
có 10 phiên là $Up$ thì KNN sẽ dự đoán đúng khoảng 4 lần và MNN sẽ dự đoán đúng 
khoảng 3 lần. Điều này có nghĩa là KNN sẽ chiếm ưu thế so MNN khi sử dụng tiêu 
chuẩn là Recall.\\\\
Xét đến Precision, ta có thể hiểu Precision như sau, với 10 lần dự đoán sẽ có 
phiên $Up$ thì KNN sẽ đúng khoảng 4 lần và MNN sẽ dự đoán đúng 6 lần. Giả sử, 
mức độ tin tưởng của chúng ta vào hệ thống là 100\%, cứ mỗi lần hệ thống dự 
đoán có phiên $Up$ thì ta sẽ quyết định đầu tư. Điều đó đồng nghĩa, nếu theo 
KNN sẽ có 6 lần ta chịu lỗ vì hệ thống dự đoán sai và với MNN thì ta sẽ có 4 
lần ta chịu lỗ.\\\\
Quay lại với Recall, giá trị này không đo đạt được việc chúng ta sẽ lợi nhuận 
hoặc thua lỗ ra sao mà thực ra là giá trị đo đạt khả năng tận dụng cơ hội của 
hệ thống.\\\\
Tới lúc này, ta có thể kết luận, bộ tiêu chuẩn chiếm ưu thế cao hơn sẽ là Accuracy 
và Precision. Điều đó cũng có nghĩa là giải thuật MNN cho kết quả ý nghĩa hơn so 
với KNN và sẽ được lựa chọn để giải quyết bài toán dự đoán xu hướng giá trị BTC.