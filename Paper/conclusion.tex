\section{Kết luận và hướng phát triển}
\subsection{Kết luận}
Kết thúc đề tài, sản phẩm cuối cùng được hoàn thiện là một công cụ nền Web hỗ 
trợ, cung cấp các thông tin có giá trị tham khảo để đầu tư Bitcoin. Dựa trên 
các con số lý thuyết, khả năng dự đoán chính xác là rất khả quan và đặc biệt, 
giải thuật được tối ưu cho phù hợp với góc nhìn của một người đầu tư.\\\\
Với không nhiều sai lệch khi so sánh bên cạnh các con số lý thuyết, khi hệ thống được cho chạy 
thực tế trong vòng 4 ngày liên tiếp (Cụ thể từ 22:30:00 13/11/2016 đến 
20:30:00 17/11/2016) đã cho ra kết quả:\\
\begin{table}[h]
\centering
\fontsize{8}{9}\selectfont
\begin{tabular}{ |c|c|c| }
\hline
Accuracy & Precision & Recall \\
\hline
64.4\% & 77.6\% & 45.5\% \\
\hline
\end{tabular}
\caption{Bảng đánh giá hệ thống thực tế }
\end{table}\\
Các tham số đánh giá chạy thực tế như vậy, có thể thấy với một lần đầu tư 
ta có tới hơn 70\% là có lợi nhuận. Tuy vậy, bất kỳ một hệ thống cũng vẫn 
sẽ có những điểm thiếu sót.\\\\
Vì giới hạn của thời gian thực hiện đề tài, phạm vi của đề tài cũng được thu 
hẹp để phù hợp nên vì thế đã bỏ qua một số yếu tố thị trường ảnh hưởng khá lớn 
đối với hướng giải quyết. Trong lúc này, bản thân có thể nhận ra hai vấn đề:
\begin{itemize}
\item Phí giao dịch: ở tất cả các sàn giao dịch, đều có một khoảng phí trung gian 
từ 0.1\% đến 0.3\% và phí này được trừ trực tiếp vào các giao dịch. Hướng tiếp 
cận của đề tài bỏ qua hoàn toàn yếu tố này và có thể hiểu là phí bằng 0\%
\item Biên độ lợi nhuận và thua lỗ: chúng ta cũng đã bỏ qua yếu tố này, mặc dù 
dựa theo đánh giá thì số lần đầu tư lợi nhuận sẽ nhiều hơn thua lỗ. Nhưng, chúng 
ta không thể kết luận việc đầu tư sẽ chắc chắn đem về lợi nhuận. Hãy nói đến một 
trường hợp xấu, biên độ lợi nhuận chỉ có \$1 cho mỗi lần nhưng biên độ thua lô 
lại là \$100, tại đây chúng ta có thể thấy là việc đầu tư không hề có lợi.
\end{itemize}
Việc nhìn nhận được các vấn đề trên không hẳn là điều tồi tệ, mà ngược lại giúp 
chúng ta có thể hiểu rõ bài toán và đưa ra những hướng phát triển tiếp theo.
\subsection{Hướng phát triển}
Với các vấn đề còn tồn tại được nêu ra bên trên (Mục 5.1), giai đoạn tiếp theo 
của đề tài là đi giải quyết vẫn đề tài như hiện giờ nhưng thêm vào đó là yếu 
tố phí giao dịch. Tuy là một yếu tố nhỏ nhưng nó dẫn đến việc thay đổi hoàn toàn 
bộ dữ liệu ban đầu, điều này đồng nghĩa toàn bộ hệ thống hiện giờ sẽ không 
tương thích. Vì thế, cần thực hiện lại quá trình xây dựng giải thuật từ đầu.\\\\
Mặc khác, việc chỉ học duy nhất từ tập dữ liệu về giá BTC là không đủ để 
đưa ra một dự đoán chính xác cao. Ngày nay, mạng xã hội đang phát triển như vũ 
bão, đây là một kênh thông tin cực kỳ quý giá, chính vì vậy mà hệ thống ở giai 
đoạn phát triển tiếp theo sự tận dụng tài nguyên này.\\\\
Phát triển hệ thống xử lý ngôn ngữ tự nhiên, xây dựng hệ thống lắng nghe các 
thông tin tài chính, chính trị có ảnh hưởng tới giá trị BTC, phân tích, 
đánh giá và cho cân bằng với hệ thống học từ dữ liệu giá BTC để cho ra một 
dự đoán tổng quát và chính xác hơn.\\\\
Đồng thời, hệ thống có thể mở rộng ra cho nhiều loại tiền mã hóa khác như: 
Ethereum, Zcash, Monero...