\section{Giới thiệu đề tài}
Bitcoin - một hệ thống tiền mã hóa (hay tiền điện tử) được xuất hiện lần đầu tiên 
vào năm 2009 bởi Satoshi Nakamoto \cite{BitcoinPaper}, với những đặc tính ưu việt hơn cả tiền tệ 
truyền thống hiện nay đã khiến cho sự tăng lên nhanh chóng về giá trị. Nhận thấy 
được sức mạnh của tiền mã hóa có thể sẽ là tương lai của kinh tế và chính trị 
nên việc hiểu rõ cũng như đầu tư vào Bitcoin là việc đáng để quan tâm.\\\\
Trong giai đoạn hiện nay, đối với nước ta, Bitcoin là một khái niệm mới vì thế 
mà việc đầu tư khi chưa có nền tảng kiến thức hoặc kinh nghiệm đầu tư là hết 
sức rủi ro. Nhận thấy vấn đề này, bản thân xây dựng một công cụ để cho nhà đầu 
tư có thể dựa vào như một yếu tố tham khảo tin cậy.\\\\
Trên một sàn giao dịch tiền mã hóa điển hình, quá trình mua bán BTC được chia ra 
thành các giai đoạn thời gian và được gọi là phiên giao dịch. Một phiên giao dịch 
được diễn tả bởi các giá trị điển hình như sau:
\begin{itemize}
\item Giá mở phiên: giá bán (mua) BTC của (các) giao dịch ngay tại thời 
điểm mở phiên.
\item Giá đóng phiên: giá bán (mua) BTC của (các) giao dịch tại thời điểm 
kết thúc phiên.
\item Giá cao nhất: giá bán (mua) BTC cao nhất của giao dịch trong khoảng 
thời gian mở phiên đến kết thúc phiên.
\item Giá thấp nhất: giá bán (mua) BTC thấp nhất của giao dịch trong khoảng 
thời gian mở phiên đến kết thúc phiên.
\end{itemize}
Thời gian của một phiên giao dịch thường được chọn là 5 phút, 30 phút, 1 tiếng, 2 tiếng, 
4 tiếng hoặc 1 ngày, ... 
Trong phạm vi đề tài chúng ta chọn thời gian một phiên giao dịch là 30 phút.\\\\
Vậy, bài toán cần giải quyết là đi dự đoán giá trị BTC trong phiên tiếp theo sẽ tăng 
hay giảm so với phiên hiện tại. Cụ thể, gọi $n$ là phiên hiện tại và $n_{close}$ 
là giá đóng phiên hiện tại, $(n+1)$ là phiên tiếp theo và $(n+1)_{close}$ là giá đóng 
phiên tiếp theo. Nếu $(n+1)_{close} > n_{close}$ thì giá tăng - $Up$, ngược lại thì, 
$(n+1)_{close} \leq n_{close}$ thì giá giảm - $Down$.\\\\
Sau khi cụ thể được yêu cầu bài toán, ta sẽ đi đặc tả hướng tiếp cận giải quyết 
vấn đề. Học máy là lựa chọn của đề tài này, cụ thể phương pháp giải quyết 
sẽ sử dụng giải thuật phân lớp để dự đoán nhãn của phiên giao dịch sẽ là $Up$ 
hay $Down$.